\documentclass{article}
% \documentclass{report}
\usepackage[utf8]{inputenc}

\title{Price Dynamics and Option Valuation of Leveraged Exchange Traded Funds}
\author{Siddharth Mantri, Manan Raval, Sanket Sojitra}
\date{May 2017}

\usepackage{tikz}

\usepackage{natbib}
\usepackage{graphicx}
\usepackage{amsmath}
\usepackage{tabularx}
\usepackage{tabu}
\usepackage[A4, margin=1in]{geometry}
\usepackage{hyperref, bookmark}
\usepackage{tikz}
\setlength{\parindent}{0}
\documentclass{article}
\usepackage{titlesec}

\setcounter{secnumdepth}{4}

\titleformat{\paragraph}
{\normalfont\normalsize\bfseries}{\theparagraph}{1em}{}
\titlespacing*{\paragraph}
{0pt}{3.25ex plus 1ex minus .2ex}{1.5ex plus .2ex}


\begin{document}

\maketitle

\begin{abstract}
 We know that S&P500 is the most actively traded derivative for ETFs. We price the options written on leveraged and inverse leveraged Exchange Traded Funds that track the S&P500 using the fast Fourier Transform method. We perform the calibration and error analysis on these LETFs using the Heston Model to calculate the parameters. These calibrated parameters are analyzed as we check Heston Model's efficiency. With the intention to find the parameters by adding a jump to the Heston Model, we perform the calibration and error analysis on the Bates Jump Diffusion Model. Overall, we compare the results given by the Heston Stochastic Volatility model and the Bates Stochastic Volatility Jump model and try to understand their behavior.     
\end{abstract}
\cleardoublepage
\pdfbookmark[section]{\contentsname}{toc}
\tableofcontents
\cleardoublepage
\section{Introduction}
\subsection{Motivation}
In past, the investors who wanted to leverage the returns had to either buy or sell options or futures contract in order to accomplish this objective. This is where the need of LETFs came into existence. The LETFs are aimed to track the multiple daily returns of an underlying asset. We consider LETFs tracking the S&P500 since they are the most traded LETFs.  For S&P500 index, we have a range of multiples available vary from -3 to +3. In the interesting work of Jośe Da Fonseca and Yahua Xu(2015), authors develop an empirical analysis of options written on LETFs related to the S&P500 index ( from -3 to +3). They successfully calibrated analysis on pricing errors as well as the parameters provided by different calibrations. they concluded that the Heston model works efficiently, except for the options written on LETF with multiple -3. We tried to replicate this work with an intention to check the efficiency of Heston when jumps are added.

\subsection{Aim of the Project}
Our aim is to conduct an analysis on options written on LETFs related to S&P500 index. First, we calibrate the Heston (1993) model on a complete set of option prices on LETFs for range of multiples from -3 to +3. We perform error analysis on these LETFs to check the efficiency of the Heston model. After getting proper results, we added jumps to the Heston model to find the parameters and performed calibration and error analysis on the Bates Jump Diffusion model. Lastly, we try to compare and understand the results of both the models.
\newline\newline
The paper is organized as follows. First we give an insight about the concepts we have used in our work. in the second section we discuss our calibration strategy. Then we discuss our experiments where we show the data-set and method description analysis. Finally we show our empirical results and show the plots to compare the efficiency of Heston model with and without jumps.

\section{Basic Concepts}
\subsection{Exchange Traded Funds and Leveraged ETFs}
An Exchange Traded Fund (ETF) is a security that is designed to track an index. Unlike mutual funds, an ETF is traded like common stock on a stock exchange.\citep{ref2} In the simplest terms, Exchange Traded Funds (ETFs) are funds that track indexes like the NASDAQ-100 Index, S&P 500, Dow Jones, etc. When you buy shares of an ETF, you are buying shares of a portfolio that tracks the yield and return of its native index. The main difference between ETFs and other types of index funds is that ETFs don't try to outperform their corresponding index, but simply replicate its performance. They don't try to beat the market, they try to be the market.\citep{ref4}
\newlin\newline


A leveraged exchange-traded fund (ETF) is a fund that uses financial derivatives and debt to amplify the returns of an underlying index. Leveraged ETFs are available for most indexes, such as the Nasdaq 100 and the Dow Jones Industrial Average. These funds aim to keep a constant amount of leverage during the investment time frame, such as a 2:1 or 3:1 ratio. Leveraged ETFs are typically used by traders who wish to speculate on an index, or to take advantage of the index's short-term momentum. Due to the high-risk, high-cost structure of leveraged ETFs, they are rarely used as long-term investments. In many cases, traders hold positions in leveraged ETFs for just a few days or less. Many leveraged ETFs have expense ratios of 1 percent or more.\citep{ref5}
Certainly, the most traded LETFs are those tracking the S&P500. For this index a large range of multiples is available. From a multiple of +1, whose ticker is SPY, that tracks the S&P500 and started to be traded in 1993 to multiple as small as -3 or as large as +3, LETFs related to this index constitute an active market. \citep{ref3}


\subsection{Stochastic Volatility Models}
A statistical method in mathematical finance in which volatility and codependence between variables is allowed to fluctuate over time rather than remain constant. "Stochastic" in this sense refers to successive values of a random variable that are not independent. Stochastic volatility is typically analyzed through sophisticated models, which became increasingly useful and accurate as computer technology improved. \citep{ref8}
\newline

Stochastic volatility models are those in which the variance of a stochastic process is itself randomly distributed. They are used in the field of mathematical finance to evaluate derivative securities, such as options. The name derives from the models' treatment of the underlying security's volatility as a random process, governed by state variables such as the price level of the underlying security, the tendency of volatility to revert to some long-run mean value, and the variance of the volatility process itself, among others. \citep{ref7}
\newline

The advantage of Stochastic Volatility model like Heston over something more traditional like the Black Scholes model is that the SV models introduce more random factors. In BSM, there is only one random factor, i.e, Volatility.  Given a short period of time, the Black Scholes model is very good at predicting local volatility. However, as time periods increase, the model begins to break as a single random factor cannot predict randomness over such a large period. The advantage of the Heston SV model is that it takes into account multiple factors. 
\newline

SV models generally model stock prices as follows:

\begin{equation}
    dS_t = r_t S_t dt + \sigma_t S_t dW_t
\end{equation}
$\sigma$ here depicts volatility and is itself a random process. It is modelled as:

\begin{equation}
    d\sigma_t = \lambda_t dt + \zeta_t dW^{\sigma}_t
\end{equation}
\newline

$W$ and $W^{\sigma}$ have $\rho$ correlation and the coefficents $\lambda$ and $\zeta$ are determined by the model parameters and are generally obtained by calibration. 
\newline

Examples of stochastic volatility models include the Heston model, the SABR model, the Chen model and the GARCH model.
\newline
\cleardoublepage
\subsection{Heston Model}
The Heston model is a Stochastic Volatility Model introduced in 1993. Developed by Steven Heston, this model describes the evolution of volatility of the asset. It assumes that the volatility follows a random process. Hence, it is not a deterministic model but rather a stochastic model.
The price dynamics of this model are given by

\begin{equation}
dS(t) = (r- q)S(t)dt + \sigma(t)S(t)dW^S(t)
\end{equation}
\begin{equation}
d\sigma(t) = \kappa(\theta-V(t))d(t) + \gamma\v(t)dW^\sigma(t)
\end{equation}
\begin{equation}
Cov[dW^S(t), dW^\sigma(t)] = \rho dt
\end{equation}
The Heston Model characteristic function is given by
\begin{equation}
    \phi(u,t) = E[e^{(iulog(S(t)))}|S_{0}, \sigma^2]
\end{equation}
\begin{equation}
    = e^{A(u,t)+B(u,t)+C(u,t)}
\end{equation}
where
\begin{equation}
    A(u,t) = iulog(S_t)
\end{equation}
\begin{equation}
    B(u,t) = \frac{\kappa\theta}{\sigma^2}((x_j+d_j)t - (2log(x_x)) )
\end{equation}
\begin{equation}
    C(u,t) = \frac{v_t(xj+dj)}{\sigma^2} (\frac{1-e^{d_j t}}{ 1-g_je^{d_j t}})
\end{equation}
\begin{equation}
    x_j = \kappa-\rho\sigma-\rho\sigma ui
\end{equation}
\begin{equation}
    d_j = x_{j}^2 - \sigma^2(ui - u^2)
\end{equation}
\begin{equation}
    x_x = \frac{ 1-(g_j e^{d_j t})}{ 1-g_j}
\end{equation}
\begin{equation}
    g_j = \frac{x_j + d_j}{x_j - d_j}
\end{equation}

The above characteristic function has been coded in Python to help us in our research.

For our experiments, we have solved the following optimization problem. Here $v_t$ is the instantaneous volatility. $\kappa$ is the rate of mean reversion of the instantaneous volatility. It describes the rate at which $v_t$ reverts to $\theta$ where $\theta$ is the long run average price of the volatility. $\sigma$ gives us the volatility of the volatility of the underlying. $\rho$ is the correlation between the spot price of the underlying and the volatility. 

The Heston model has some underlying advantages: 
\begin{itemize}
    \item The volatility is mean reverting
    \item It takes into account leverage effect between the returns and the volatility.
\end{itemize}

The model captures volatility clustering, mean reversion and displays heavier tails. However, it fails to capture random fluctuations in the market. An example could be a factor that causes the market to crash. The prices of the model are also very sensitive to the calibrated parameters. 

\subsection{Bates Model}
The Bates model was introduced by David Bates in 1996. It extends on the Heston SVM by  adding jumps in the price process of the underlying asset. 


The price dynamics of this model are given by
\begin{equation}
dS_t = (r- q)S_t dt + \sigma_t S_t dW^{S_t} +J_t S_t N_t
\end{equation}
\begin{equation}
d\sigma(t) = \kappa(\theta-V(t))d(t) + \gamma\v(t)dW^\sigma(t)
\end{equation}
\begin{equation}
Cov[dW^S(t), dW^\sigma(t)] = \rho dt
\end{equation}
The Heston Model characteristic function is given by
\begin{equation}
    \phi(u,t) = E[e^{(iulog(S(t)))}|S_{0}, \sigma^2]
\end{equation}
\begin{equation}
    = e^{A(u,t)+B(u,t)+C(u,t)+D(u,t)}
\end{equation}
where
\begin{equation}
    A(u,t) = iulog(S_t)
\end{equation}
\begin{equation}
    B(u,t) = \frac{\kappa\theta}{\sigma^2}((x_j+d_j)t - (2log(x_x)) )
\end{equation}
\begin{equation}
    C(u,t) = \frac{v_t(xj+dj)}{\sigma^2} (\frac{1-e^{d_j t}}{ 1-g_je^{d_j t}})
\end{equation}
\begin{equation}
    D(u,t) = (-\lambda \mu_j i u t)+\lambda t(1+\mu_j)^{iu} e^{\frac{1}{2} \sigma_j ^2 iu(iu-1)}-1
\end{equation}
\begin{equation}
    x_j = \kappa-\rho\sigma-\rho\sigma ui
\end{equation}
\begin{equation}
    d_j = x_{j}^2 - \sigma^2(ui - u^2)
\end{equation}
\begin{equation}
    x_x = \frac{ 1-(g_j e^{d_j t})}{ 1-g_j}
\end{equation}
\begin{equation}
    g_j = \frac{x_j + d_j}{x_j - d_j}
\end{equation}

In the Bates model, the original parameters of the Heston SV model behave in the same way. However, this model adds three new parameters, namely, $\lambda$, $\mu_j$ and $\sigma_j$. These three parameters control the following

\subsection{Option Pricing}
In modern financial analysis, the Black Scholes equation is an extremely well known and widely used method of pricing financial assets, especially in pricing options. Before venturing into the world of trading options, investors should have a good understanding of the factors that determine the value of an option. These include the current stock price, the intrinsic value, time to expiration or the time value, volatility, interest rates and cash dividends paid. Let's start with the primary drivers of the price of an option: current stock price, intrinsic value, time to expiration or time value, and volatility. The current stock price is fairly obvious. The movement of the price of the stock up or down has a direct - although not equal - effect on the price of the option. As the price of a stock rises, the more likely the price of a call option will rise and the price of a put option will fall. If the stock price goes down, then the reverse will most likely happen to the price of the calls and puts. \citep{ref9} 
\newline\newline
Given the success of the Black-Scholes model in parsimoniously describing market options prices, it’s not immediately obvious what the benefits of making such a modelling choice might be. Stochastic volatility models are useful because they explain in a self consistent way why it is that options with different strikes and expiration's have different Black-Scholes implied volatilities (“implied volatilities” from now on) – the “volatility smile”. In particular, traders who use the BlackScholes model to hedge must continuously change the volatility assumption in order to match market prices. Their hedge ratios change accordingly in an uncontrolled way. More interestingly for us, the prices of exotic options given by models based on Black-Scholes assumptions can be wildly wrong and dealers in such options are motivated to find models which can take the volatility smile into account when pricing these.\citep{ref6} Hence we can say that using a Stochastic Volatility model like the Heston Model can be beneficial and this is the reason we are trying to track its efficiency with jumps



\section{Model Calibration}
\subsection{Need for Calibration and Problem Formulation}
The need for developing stochastic volatility models and SV models with jumps comes from the need to fit market-available data in a better manner. Model calibration is the process of fitting a given model to available market data. Calibration methods generally employ optimization techniques which, given a set of parameters, obtain parameters that best fit and describe the observable market prices. The closer a given model gets to market data, the better its chances are to be used for risk and portfolio management. 
\newline\newline

For each LETF and leveraged ratio m, we calibrate the following volatility optimisation model.

\begin{equation}
    \min_{v_{0},\kappa,\theta,\sigma,\rho}\frac{1}{N}\sum_{i}^{N}\big(\sigma_{imp}^{market}\big(t_{i},K_{i},m\big) - \sigma_{imp}^{model}\big(t_{i},K_{i},m\big)\big)^2
\end{equation}
where $\sigma_{imp}^{market}\big(t_{i},K_{i},m\big)$ is the market implied volatility for the option with maturity $t_i$, strike $K_i$ and leverage ratio $m$. The number of options is given by N. Similarly, $\sigma_{imp}^{model}\big(t_{i},K_{i},m\big)$ is the model implied volatility. We also report the error in option pricing, which is given by
\begin{equation}
    \min_{v_{0},\kappa,\theta,\sigma,\rho}\frac{1}{N}\sum_{i}^{N}\big(c_{imp}^{market}\big(t_{i},K_{i},m\big) - c_{imp}^{model}\big(t_{i},K_{i},m\big)\big)^2
\end{equation}
where $c_{imp}^{market}\big(t_{i},K_{i},m\big)$ is the market price with maturity $t_i$, strike $K_i$ and leverage ratio $m$.

\cleardoublepage
\section{Experiment}
\subsection{Dataset Description}
For this experiment, we have a rich dataset of 6 Leveraged ETFs tracking the S&P 500. The dataset is described in the table below. 

\begin{table}[h!]
\centering
\caption{Data Description}
\label{my-label}
\begin{tabular}{|l|l|l|l|l|l|}
\hline
Tickers & Leverage Ratio & Number of options & Smallest Maturity & Largest Maturity & Moneyness Range \\ \hline
SPXU & -3 & 330 & 0.024657534 & 1.769863014 & {[}0.05733945, 3.440366972{]} \\ \hline
SDS & -2 & 352 & 0.016438356 & 1.783561644 & {[}0.07423905, 2.598366741{]} \\ \hline
SH & -1 & 334 & 0.0219 & 0.769863014 & {[}0.516499283, 1.721664275{]} \\ \hline
SPY & 1 & 3171 & 0.0054 & 2.695890411 & {[}0.340136054, 1.445578231{]} \\ \hline
SSO & 2 & 1032 & 0.0198 & 1.7452 & {[}0.357568534, 1.549463647{]} \\ \hline
UPRO & 3 & 1322 & 0.0082 & 0.778082192 & {[}0.260443796, 1.458485259{]} \\ \hline
\end{tabular}
\end{table}

Downloaded from the delayed quotes table of the Chicago Board Options Exchange (CBOE), we conducted the following steps for each LETF. 

\begin{enumerate}
    \item Download Options and LEAPs data from CBOE
    \item Clean out LEAPs and options which have a bid of less than 0.5
    \item The data downloaded gave us only the Ticker, Option type, bid and ask prices and the current stock price of the underlying. 
    \item Added the following data:
    \begin{itemize}
         \item Strike, by getting the data from the ticker symbol.
         \item Maturity date, again by deciphering the ticker symbol for an option
         \item Moneyness, calculated using Strike/Stock
         \item Market price, calculated using the average of the given bid and ask prices
    \end{itemize}
    \item The data was then converted to a dictionary format to be used with ease in our Python code
\end{enumerate}


\subsection{Method}

For us to numerically implement the Heston model, we had to study the paper by Carr and Madan[1999]
//TODO:pseudocode for Heston, Bates and Calibration
//TODO:No need for plot generation pseudo code

\section{Results}
\subsection{Calibration Results for Heston Model}
The Heston Model has 5 parameters we must estimate: $\kappa, \sigma, \theta, \rho$ and $v_t$. We then analyze the calibration parameters on each LETFs and calculated the errors in volatility and prices for these LETFs which is shown in the table below:

\begin{table}[h!]
\centering
\caption{Heston Model Calibration Result}
\label{my-label}
\begin{tabular}{|l|l|l|l|l|l|l|l|}
\hline
{Ticker} & {v0} & {$\kappa$}  & {$\theta$}  & {$\sigma$}  & {$\rho$}   & {Volatility Error} & {Price Error} \\ \hline
\textbf{SPXU}   & 0.414680147 & 2.892590558 & 0.051511783 & 3.545262624 & -1.15E-39    & 0.004218577               & 0.00063749           \\ \hline
\textbf{SDS}    & 0.334578367 & 0.816575627 & 0.306771304 & 4.511649717 & -0.759914518 & 0.012037415               & 0.000771869          \\ \hline
\textbf{SH}     & 0.042355185 & 9.852268012 & 0.048968957 & 7.475109789 & -0.041592367 & 0.004040145               & 0.000470654          \\ \hline
\textbf{SPY}    & 0.188624474 & 0.377395699 & 0.52472413  & 9.999827962 & -0.519070198 & 0.000101711               & 1.71E-05             \\ \hline
\textbf{SSO}    & 0.220656022 & 1.077617466 & 0.260470715 & 4.721484288 & -0.811187715 & 0.001104727               & 0.000104577          \\ \hline
\textbf{UPRO}   & 0.138642899 & 0.118229441 & 0.135372546 & 0.87355137  & -0.63976789  & 0.003144009               & 0.000563037          \\ \hline
\end{tabular}
\end{table}

//TODO:Here we explain results of calibrated parameters

//TODO:Table with calibrated parameters along with price and vol errors

//TODO:Images of all results along with errors

//TODO:explain results and expected numbers and unexpected numbers

\subsection{Calibration Results for Bates Model}

\section{Conclusion}
%\section{Bibliography}
\section{Code}


\bibliographystyle{plain}
\bibliography{references}
\citep{ref1}
\citep{ref2}
\citep{ref3}
\citep{ref4}
\citep{ref5}
\citep{ref6}
\citep{ref7}
\citep{ref8}
\citep{ref9}

\end{document}
